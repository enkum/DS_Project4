% Options for packages loaded elsewhere
\PassOptionsToPackage{unicode}{hyperref}
\PassOptionsToPackage{hyphens}{url}
%
\documentclass[
]{article}
\title{R Notebook}
\author{}
\date{\vspace{-2.5em}}

\usepackage{amsmath,amssymb}
\usepackage{lmodern}
\usepackage{iftex}
\ifPDFTeX
  \usepackage[T1]{fontenc}
  \usepackage[utf8]{inputenc}
  \usepackage{textcomp} % provide euro and other symbols
\else % if luatex or xetex
  \usepackage{unicode-math}
  \defaultfontfeatures{Scale=MatchLowercase}
  \defaultfontfeatures[\rmfamily]{Ligatures=TeX,Scale=1}
\fi
% Use upquote if available, for straight quotes in verbatim environments
\IfFileExists{upquote.sty}{\usepackage{upquote}}{}
\IfFileExists{microtype.sty}{% use microtype if available
  \usepackage[]{microtype}
  \UseMicrotypeSet[protrusion]{basicmath} % disable protrusion for tt fonts
}{}
\makeatletter
\@ifundefined{KOMAClassName}{% if non-KOMA class
  \IfFileExists{parskip.sty}{%
    \usepackage{parskip}
  }{% else
    \setlength{\parindent}{0pt}
    \setlength{\parskip}{6pt plus 2pt minus 1pt}}
}{% if KOMA class
  \KOMAoptions{parskip=half}}
\makeatother
\usepackage{xcolor}
\IfFileExists{xurl.sty}{\usepackage{xurl}}{} % add URL line breaks if available
\IfFileExists{bookmark.sty}{\usepackage{bookmark}}{\usepackage{hyperref}}
\hypersetup{
  pdftitle={R Notebook},
  hidelinks,
  pdfcreator={LaTeX via pandoc}}
\urlstyle{same} % disable monospaced font for URLs
\usepackage[margin=1in]{geometry}
\usepackage{graphicx}
\makeatletter
\def\maxwidth{\ifdim\Gin@nat@width>\linewidth\linewidth\else\Gin@nat@width\fi}
\def\maxheight{\ifdim\Gin@nat@height>\textheight\textheight\else\Gin@nat@height\fi}
\makeatother
% Scale images if necessary, so that they will not overflow the page
% margins by default, and it is still possible to overwrite the defaults
% using explicit options in \includegraphics[width, height, ...]{}
\setkeys{Gin}{width=\maxwidth,height=\maxheight,keepaspectratio}
% Set default figure placement to htbp
\makeatletter
\def\fps@figure{htbp}
\makeatother
\setlength{\emergencystretch}{3em} % prevent overfull lines
\providecommand{\tightlist}{%
  \setlength{\itemsep}{0pt}\setlength{\parskip}{0pt}}
\setcounter{secnumdepth}{-\maxdimen} % remove section numbering
\ifLuaTeX
  \usepackage{selnolig}  % disable illegal ligatures
\fi

\begin{document}
\maketitle

Food and housing insecurity is a social problem. This is an interesting
research we loved from the get go. One of the problems we encountered
was the missing values in the data, and how to handle them, especially a
question that was a dependent variable and had about 95\% of the data
missing. We also faced some challenges when it came to questions with
multiple choices and how to compound them. We began the data cleaning
with Excel which was counter reproducible. Because of what we have
learnt in this course, we began the data cleaning again using R
programming language, which makes the data cleaning process
reproducible{[}1{]}.

Considering one of the dependent variables, we were caught between
making another choice for the response variable and using machine
learning algorithm for missing value imputation because about 95\% of
the data was missing. We had to schedule a meeting with the collaborator
in other express our concerns and propose our solutions. This meeting
helped us understand the best approach to handle the missing value in
the data.

Also we exchanged our codes and data through outlook, which made the
whole collaborative process strenuous. But after the class where we
discuss collaborative web services like GitHub and binder, we began to
save our codes on github and share it among ourselves which makes it
easily editable therefore making collaboration much easier.{[}1{]}

Reference

1.{[}Roger D Peng{]}Reproducible research checklist
\url{https://www.youtube.com/watch?v=pjL2uP-fmPY\&list=PLjTlxb-wKvXOU4WW4p3qc4VKWTI4gLNUf\&index=4}

2.Link to github files
\url{https://github.com/fbiney/DS_Project4/network/members}

\end{document}
